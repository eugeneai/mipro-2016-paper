\documentclass[a4paper,12pt]{article}

\usepackage{amssymb,amsfonts,graphics, graphicx, amsthm, amscd}
\usepackage[cmex10]{amsmath}
\usepackage[utf8]{inputenc}
\usepackage[english,russian]{babel}

\usepackage{enumerate}
\usepackage{indentfirst}
\usepackage{color}
\usepackage{algorithmic}
\usepackage{algorithm}
\usepackage{float}
\usepackage{tikz}
\usepackage{tikz-qtree}

\tolerance=500

\headheight=0mm
\headsep=0mm
\textwidth=180mm
\textheight=250mm
\topmargin=-10mm
\oddsidemargin=-10mm
\evensidemargin=-10mm


\title{Positive Constructed Formulas Preprocessing\\ for Automatic Deduction\thanks{RSCF}}
\author{Artem~Davydov, Alexander~Larionov, Evgeny~Cherkashin}
\date{}

%%% \author{\IEEEauthorblockN{Artem~Davydov${}^{*}$, Alexander~Larionov${}^{*}$ and Evgeny Cherkashin${}^{*\;**}$} \IEEEauthorblockA{${}^{*}$Institute of System Dynamics and Control Theory at Siberian Branch of Russian Academy of Sciences, \\ ${}^{**}$National Research Irkutsk State Technical University,\\ Irkutsk, Russian Federation\\ \texttt{\{mikhailov,alex,eugeneai,bychkov\}@icc.ru}}}

\begin{document}

\maketitle

\begin{abstract}
The preprocessing of positively--constructed formulas (PCFs) for automatic deduction search algorithms is considered in this paper.  A new efficient algorithm for conversion of the predicate calculus language formulas to the language of PCFs and equivalent rules of reducing PCFs, preserving the original heuristic structure of knowledge represented by the formulas of the predicate calculus language, are presented.
\end{abstract}


\newtheorem{definition}{Definition}
\newtheorem{example}{Example}

\newcommand{\fictAquantor}{\ensuremath{\forall\colon\varnothing}}
\newcommand{\fictEquantor}{\ensuremath{\exists\colon\varnothing}}
\newcommand{\bomega}{\boldsymbol{\omega}}
\newcommand{\bphi}{\boldsymbol{\phi}}
\newcommand{\eqdef}{\stackrel{\mathrm{df}}{=}}
\newcommand{\bigand}[2]{\raisebox{-2pt}{\ensuremath{\overset{#1}{\underset{#2}{\text{\Large\&\normalfont}}}}}}



\section{Introduction}

Originally \cite{SNV1990,ICDS2000} the calculus of positively constructed formulas (PCF) was developed by Russian scientists S.N. Vassilyev and A.K. Zherlov by an evolutionary way in describing and solving control theory (CT) problems. In \cite{ICDS2000} the PCF calculus is presented as first-order logical formalism (further development presendted in \cite{jour2} ), examples of CT problems described and solved by the PCF calculus (elevator group control, mobile robot action planning and telescope guidance), as well as proof of soundness and completeness.

The PCF calculus is both machine-oriented, and also human-oriented, naturally aimed at solving the problems of dynamic systems control thanks to its features such as follows: unique inference rule and simple scheme of axioms; modifyability of semantics (constructive, monotonic, temporal, etc.) and besides it is possible to construct intuitionistic inferences of some non-Horn formulas; the explicit usage of $\forall$-- and $\exists$--quantifiers, the scolemization procedure is not required.

However, in this paper, we will not cover the human--orientation properties of the PCF calculus and its dynamic systems control properties as well.  This and the calculus capabilities in action planning is described in \cite{ICDS2000}.  The problems of an automatoc theprem proving software (provers) design and implementation are considered briefly in \cite{mipro2013}.

We describe the language of PCF's, conversion of first--order predicate calculus formulas (FOF) to a PCF's form and the preprocessing of PCFs for automatic deduction search algorithms.  The conversion is not a simple problem.  The resulting PCF's must obey the main qualitative criteria that the conversion should preserve the heuristic structure of the original formulas.  The heuristic structure reflects the properties of the problems being modeled with the logical calculus.  So the preservation the structure during conversion allows one to pass the problem properties to the inference engine, resulting in improving performance of the inference search by means application of the so called domain--spec strategies.  The proposed new algorithm of conversion of FOF into PCF's improve the heuristic preservation quality as compared to the our early used \cite{mipro2013} and developed by A.~K.~Zherlov and E.~A.~Cherkashin.

%Here, we deal with the automatic search of a logical inference, i.e., the inference engine capabilities.  In order to estimate the applicability of the PCF calculus for automatic theorem proving we develop a prover program and test it on problems from TPTP library.

%This paper contains a description of the PCF calculus, an implementation of the prover program, and strategies of logical inference used to direct the inference search algorithms. The results of a comparison of the developed prover system with other provers are presented.


%=======================================================================
%==========================BACKGROUND===================================
%=======================================================================
\section{Preliminaries}

Let's consider the basic ideas behind PCF's and its calculi.  The language of PCF is a restricted variant of the language of first-order logic (FOL), which consists of first--order formulas (FOFs) built out of atomic formulas with $\&, \vee, \neg, \rightarrow, \leftrightarrow$ operators, $\forall$ and $\exists$ quantifier symbols and constants $\text{\textbf{True}}$ and $\text{\textbf{False}}$.  The concepts of \emph{term}, \emph{atom}, \emph{literal} we define in the usual way.

Let $X = \{x_1,\ldots,x_k\}$ be a set of variables, $A = \{A_1,\ldots,A_m\}$ be a set of atomic formulas, and $F = \{F_1,\ldots,F_n\}$ be a set of subformulas. Then the following formulas $((\forall x_1) \ldots (\forall x_k) (A_1 \& \ldots \& A_m \rightarrow (F_1 \vee \ldots \vee F_n)))$ and $((\exists x_1) \ldots (\exists x_k) (A_1 \&$ $\ldots \& A_m \& (F_1 \& \ldots \& F_n)))$ are denoted as  $\forall_XA\colon F$ and $\exists_XA\colon F$ respectively, keeping in mind that the $\forall$--quantifier corresponds to $\rightarrow F^{\vee}$, where $F^{\vee}$ means disjunction of all subformulas from $F$, and $\exists$--quantifier corresponds to $\& F^{\&}$, where $F^{\&}$ means conjunctions of all subformulas from $F$.

If $F = \varnothing$, then the formulas have the form $\forall_XA\colon\varnothing \equiv \forall_XA \rightarrow \text{\textbf{False}}$ and $\exists_XA\colon\varnothing \equiv \exists_XA \& \text{\textbf{True}}$, since the empty disjunction is identical to $false$, whereas the empty conjunction is identical to $\text{\textbf{True}}$.  The form $\forall_XA$ and $\exists_XA$ are abbreviations of such formulas.  If $X = \varnothing$, then $\forall A\colon F$ and $\exists A\colon F$ are analogous abbreviations.

The set of atoms $A$ is called {\em conjunct}. The empty conjunct is identical to $\text{\textbf{True}}$ as it was already mentioned.

Variables from $X$ are bound by corresponding quantifiers and called $\forall$--variables and $\exists$--variables, respectively. In $\forall_XA$, a variable from $X$ that does not appear in conjunct $A$ is called {\em unconfined} variable.

%В связи с изложенными сокращениями отметим следующий факт:
$\forall \varnothing \equiv \forall \varnothing\colon\varnothing \equiv \forall \text{\textbf{True}} \rightarrow \text{\textbf{False}} \equiv \text{\textbf{False}}$

Construction $\forall_XA$ and $\exists_XA$ are called positive \emph{type quantifiers} (TQ), because $A$ is a conjunction of only positive atoms and referred to as also as \emph{type condition} for $X$. In practice, this constructions denote phrases such as follows: ``for all $X$ satisfying $A$ there is...'' or ``there exists $X$ satisfying property $A$ such that...''; ``for all integer $x,y,z$ and $n>2$ there is $x^n + y^n \ne z^n$''.

Originally, the term ``type quantifier'' was introduced by N.~Bourbaki \cite{Bourbaki} as part of notation for formalization of mathematics. But type quantifiers are stable in languages of another applied fields.

%========================================================
\subsection{PCF Language Explicit Definition}

\begin{definition}[Positively--constructed formulas (PCF)]
\label{def:pcf}
Let, $X$ be a set of variables, and $A$ be a conjunct.
\begin{enumerate}

\item $\exists_XA$ and $\forall_XA$ are $\exists$--PCF and $\forall$--PCF respectively.

\item If $F = \{F_1,\ldots,F_n\}$ are $\forall$--PCF, then $\exists_XA\colon F$ is a $\exists$--PCF.

\item If $F = \{F_1,\ldots,F_n\}$ are $\exists$--PCF, then $\forall_XA\colon F$ is a $\forall$--PCF.

\item Any $\exists$--PCF or $\forall$--PCF is a PCF.
\end{enumerate}
\end{definition}

This form of logical formulas is referred to as positively constructed formulas (PCFs), as they are written with only positive type quantifiers. The formulas contain no explicit logic negation sign. Any FOF can be represented as PCF \cite{ICDS2000}.
%Таким образом ПО--формула есть особый вид записи классических формул языка предикатов, подобно КНФ, ДНФ и др.

PCF starting from $\forall \varnothing$ is called PCF in a {\em canonical form}. Any PCF can be represented in the canonical form. Let $F$ is a non--canonical $\exists$--PCF, then the canonical PCF
$\forall \varnothing\colon F \equiv \text{\textbf{True}} \rightarrow F \equiv F$. If a PCF $F$ is a non--canonical $\forall$--PCF, then the canonical PCF $\forall \varnothing\colon\{\exists \varnothing\colon F\} \equiv \text{\textbf{True}} \rightarrow \{\text{\textbf{True}}\&F\} \equiv F$.  Type quantifiers $\forall \varnothing$ and $\exists \varnothing$ are called {\em fictitious}, since they do not influence truth value of an original PCF and do not bind any variables.  Thy are used to regularize PCFs to a canonical ones.

%Для удобства читаемости формул, будем представлять их в древовидной форме следующим образом:
The PCFs are usually represented as trees for more ease reading, i.e. $Q_XA\colon\{F_1,\ldots,F_n\}$ is represented as
\begin{center}
\begin{tikzpicture}
\tikzset{grow'=right}
%\tikzset{execute at begin node=\strut}
\tikzset{level distance=80pt}
\tikzset{every tree node/.style={align=left,anchor=west}}
\Tree [. $Q_XA$ $F_1,$ $\cdots$ $F_n;$ ]
\end{tikzpicture}
\end{center}
\noindent where $Q$ is a quantifier. Tree elements have conventional names: \emph{node}, \emph{root}, \emph{leaf}, \emph{branch}, etc. As the quantifiers $\forall$ correspond to disjunctions of formulas $\{F_1,\ldots,F_n\}$ (quantifiers $\exists$ correspond to  conjunctions), then each $\forall$--node is considered to be a {\em disjunctive branching}, and each $\exists$--node is correspond to be a {\em conjunctive branching}.

Some parts of canonical PCF are denoted as follows:
\begin{enumerate}
\item PCF root $\forall \varnothing$ is called PCF {\em root}.
\item Each PCF root child $\exists_XA$ is called PCF {\em base}, conjunct $A$ is called {\em base of facts}, and PCF rooted from base is called {\em base subformula}.
\item PCF base children $\forall_YB$ are called {\em questions} to its parent base.  If a question is a leaf of a tree then it is called {\em goal question}.
\item Subtrees of questions are called {\em consequents}.  If a question has no consequent then the question is referred to as \emph{goal question}, and it is identical to $\text{\textbf{False}}$.
\end{enumerate}

%как правило
%In the sequel, only PCFs in the canonical form are considered.
%----------------------

\begin{example}
Let us consider a PCF representation of a FOF
$$F= \neg\bigl(\forall x\:\exists y P(x,y)\rightarrow \exists z P(z,z)\bigr).$$
\end{example}
An image $F'$ of $F$ in the PCF language is $F' = \forall\colon \varnothing\{\exists\colon\varnothing\{\forall x\colon\varnothing\{\exists y\colon P(x,y)\}, \forall z\colon P(z,z)\{\exists\colon\boldsymbol{False}\}\}\}.$
%А эта ПО-формула в древовидной форме имеет следующий вид+++++++++++++++++++++++++
The tree--like form of the latter is as follows:
\begin{center}
\begin{tikzpicture}
\tikzset{grow'=right}
%\tikzset{execute at begin node=\strut}
\tikzset{level distance=80pt}
\tikzset{every tree node/.style={align=left,anchor=west}}
\Tree [. $\forall\colon\varnothing$ [. $\exists\colon\varnothing$ [. $\forall x\colon\varnothing$ $\exists y\colon P(x,y),$ ] [. $\forall z\colon P(z,z)$ $\exists\colon\boldsymbol{False}.$ ] ] ]
\end{tikzpicture}
\end{center}

\section{The transformation algorithm}

% Этот параграф посвящен изложению основного результата работы - алгоритма преобразования формул языка ИП в язык ПОФ.

The algorithm's input is a FOF.  At the first stage, the algorithm performs some general transformations:
\begin{itemize}
\item elimination of $\leftrightarrow$ and $\rightarrow$ logical connections according to well known equivalent transformations $F_1\leftrightarrow F_2 = (F_1\rightarrow F_2)\&(F_2\rightarrow F_1)$ and $F_1\rightarrow F_2 = \neg F_1 \vee F_2$;
\item application of DeMorgan laws $\neg (F_1\&\ldots\& F_n)=\neg F_1\vee\ldots\vee\neg F_n$, $\neg (F_1\vee\ldots\vee F_n)=\neg F_1\&\ldots\&\neg F_n$, $\neg\forall x F = \exists x \neg F$, $\neg\exists x F = \forall x \neg F$ and elimination of consequent negations $\neg\neg F = F$; these transformations shift connections $\neg$ to leaves of the FOF (atoms).
\end{itemize}
The above mentioned transformations will break the original heuristic structures, which was given in the FOF, however, reduction rules described below will reconstruct some of the heuristics to make resulting PCF look similar to the original FOF.  Elimination of the connections $\leftrightarrow$ and $\rightarrow$ allows us to define the conversion algorithm laconically as a tree--like FOF syntactic representation transformations.

Let $F$ be a FOF processed according above mentioned transformations.  A \emph{syntactic tree} of a FOF $F$ is a tree whose root is either
\begin{enumerate}
\item the formula $F$ itself, if $F$ a literal (an atom or an atom negation).
\item a quantifier $\forall x$ or $\exists x$, if $F$ is of a form $\forall x G$ or $\exists x G$ correspondingly; syntactic tree of $G$ is the only subtree of syntactic tree of $F$.
\item a logical connection $\circ$, if $F$ is of a form $G_1\circ\ldots\circ G_n$, where $\circ$ is connection $\&$ or $\vee$, all subtrees of this node are syntactic trees of FOF's $G_i, \forall i=\overline{1,n}$ correspondingly.  The connection $\&$ in a FOF as usual has higher priority than $\vee$, the brackets are used to adjust the priorities (as usual); e.g., if a FOF $\neg A \& B \vee C$, there are two subformulas $G_1 = \neg A\& B$ и $G_2 = C$.
\end{enumerate}
Таким образом построенное дерево листами имеет атомы или отрицания атомов. Для каждого узла $N$ дерева $T$, $G_N$ будет обозначать подформулу формулы $F$, соответствующую поддереву $T'$ с корнем в узле $N$. $N'_i,\forall i=\overline{1,n}$ будет обозначать непосредственно следующие за $N$ узлы.

Для примера подготовки формулы к работе алгоритма рассмотрим следующую: $$F = \forall x\forall y(S(x,y)\leftrightarrow\forall z(I(z,x)\rightarrow I(z,y))).$$
Устраняем связки $\leftrightarrow$ и $\rightarrow$.
$$\begin{array}{l}
F = \forall x\forall y( F_1 \& F_2 )\\
F_1 = \neg S(x,y)\vee\forall z(\neg I(z,x)\vee I(z,y))\\
F_2 = \neg\forall z(\neg I(z,x)\vee I(z,y)) \vee S(x,y) = \exists z( I(z,x)\&\neg I(z,y) ) \vee S(x,y)
\end{array}
$$

Синтаксическое дерево для $F$:
\begin{center}
\Tree[. $\forall x$ [. $\forall y$ [. $\&$ $F_1$ $F_2$ ] ] ]
\Tree[. \hspace{5mm} ]
\Tree[. $F_1$ [. $\vee$ $\neg S(x,y)$ [. $\forall z$ [. $\vee$ $\neg I(z,x)$ $I(x,y)$ ] ] ] ]
\Tree[. $F_2$ [. $\vee$ [. $\exists z$ [. $\&$ $I(z,x)$ $\neg I(z,y)$ ] ] $S(x,y)$ ] ]
\end{center}


Введем еще одно обозначение. Пусть $P,Q\in\{\forall,\exists\}, P\neq Q$, $A$ - некоторый конъюнкт, тогда:
\begin{displaymath}
\begin{array}{l}
%A=A^{\&} = \bigand{n}{i=1}A_i(t_1,\ldots,t_k)\\
F^Q = \left\lbrace
		  \begin{array}{l}
		  F, \text{если } F = Q_X\colon A \;\Phi,\\
		  Q\colon\varnothing(F), \text{если } F = P_X\colon A \;\Phi,
		  \end{array}\right.

\end{array}
\end{displaymath}

\floatname{algorithm}{Алгоритм}
\renewcommand{\algorithmicrequire}{\textbf{Input:}}
\renewcommand{\algorithmicensure}{\textbf{Output:}}
\begin{algorithm}
\caption{ Преобразования формул языка ИП к виду ПОФ.}

The rules of transformation are as follows:


\begin{algorithmic}
\REQUIRE Корневая вершина $N$ cинтаксического дерева $T$ для $F$.
\ENSURE $F^{\pi}$ - ПОФ-представление формулы $F$.
\IF {$N = Qx$}
  \RETURN  $F^{\pi} = Qx\colon\varnothing ( (G_{N'}^{\pi})^{P} )$   \COMMENT {$P\neq Q$}
\ENDIF
\IF {$N=\vee$}
  \RETURN  $F^{\pi} = \forall\varnothing \bigl( (G_{N'_1}^{\pi})^{\exists},\ldots,(G_{N'_k}^{\pi})^{\exists}\bigr)$
\ENDIF
\IF {$N=\&$}
  \RETURN  $F^{\pi} = \exists\varnothing \bigl( (G_{N'_1}^{\pi})^{\forall},\ldots,(G_{N'_k}^{\pi})^{\forall}\bigr)$
\ENDIF
\IF[$R$ - некоторый атом] {$N=R$}
  \RETURN  $F^{\pi} = \exists\varnothing\colon R$
\ENDIF
\IF[$R$ - некоторый атом] {$N=\neg R$}
  \RETURN  $F^{\pi} = \forall\varnothing\colon R$
\ENDIF
\end{algorithmic}
\end{algorithm}
Очевидно, что данный алгоритм всегда заканчивает свою работу за конечное число шагов, равное количеству узлов синтаксического дерева для исходной формулы $F$, и выдает корректный результат, ПОФ-представление формулы $F$, так как на каждом шаге алгоритма каждый узел синтаксического дерева преобразуется к виду ПОФ.

Преобразованная формула из предыдущего примера будет выглядеть следующим образом.

\begin{center}
\begin{tikzpicture}
\tikzset{grow'=right}
%\tikzset{execute at begin node=\strut}
\tikzset{level distance=50pt}
\tikzset{every tree node/.style={align=left,anchor=west}}
\Tree [. $\forall x\colon\varnothing$ [. $\exists\varnothing$ [. $\forall y\colon\varnothing$ [. $\exists\varnothing$
[. $\forall\varnothing$ [. $\exists\varnothing$ $\forall\colon S(x,y)$ ]
			[. $\exists\varnothing$ [. $\forall z\colon\varnothing$ [. $\exists\varnothing$  $\forall\colon I(z,x)$ ] $\exists\colon I(x,y)$  ] ] ]
[. $\forall\varnothing$ [. $\exists z\colon\varnothing$ [. $\forall\varnothing$ $\exists\colon I(z,x)$ ] $\forall\colon I(z,y)$ ] $\exists\colon S(x,y)$ ] ] ] ] ]
\end{tikzpicture}
\end{center}

Как видно, при работе алгоритма в получаемой формуле возникает слишком много узлов с фиктивными кванторами $\forall\varnothing$ и $\exists\varnothing$, кроме этого, могут возникнуть излишние ветвления, плохо влияющие на дальнейший поиск вывода. Следующее правило позволяет максимально сократить получаемое дерево.

\textbf{Теорема. (Правило сокращения)}

Если в некоторой ПО-формуле $F$:
\begin{center}
\begin{tikzpicture}
\tikzset{grow'=right}
%\tikzset{execute at begin node=\strut}
\tikzset{level distance=60pt}
\tikzset{every tree node/.style={align=left,anchor=west}}
\Tree [. $Q_X\colon A$ [. $P_Y\colon B$ [. $Q\colon C$ [. $P_{Z_1}\colon C_1$ $\Phi_1$ ] $\cdots$ [. $P_{Z_k}\colon C_k$ $\Phi_k$ ] ] $\Psi$ ] $\Phi$ ]
\end{tikzpicture}
\end{center}
$P,Q\in\{\forall,\exists\}, P\neq Q$, $C,B$ - некоторые конъюнкты, удовлетворяющие условию $C\subseteq B$, то $F$ эквивалентна $F'$ имеющей следующий вид:

\begin{center}
\begin{tikzpicture}
\tikzset{grow'=right}
%\tikzset{execute at begin node=\strut}
\tikzset{level distance=80pt}
\tikzset{every tree node/.style={align=left,anchor=west}}
\Tree [. $Q_X\colon A$ [. $P_{Y,Z_1}\colon B,C_1$ $\Phi_1\cup\Psi$ ] $\cdots$ [. $P_{Y,Z_k}\colon B,C_k$ $\Phi_k\cup\Psi$ ] $\Phi$ ]
\end{tikzpicture}
\end{center}

\textbf{Доказательство.}

Пусть конъюнкт $C$ имеет вид $C^1\&\ldots\& C^n$. Если $C\subseteq B$, то $B = B'\&C^1\&\ldots\& C^n$, где $B'$ некоторый, возможно пустой конъюнкт. Переведем ПО-формулу $F$ в язык исчисления предикатов. В случае, если $F$ начинается с квантора $\forall$:
$$F = \forall_X A\rightarrow\bigl(\Phi\vee\exists_Y(\Psi\& B\&(\neg C\vee\exists_{Z_1}(C_1\&\Phi_1)\vee\ldots\vee\exists_{Z_k}(C_k\&\Phi_k)))\bigr)$$
При раскрытии скобок получаем дизъюнкицию конъюнкций, одна из которых: $\Psi\&B\&\neg C = (\Psi\&B'\&C^1\&\ldots\& C^n)\&(\neg C^1\vee\ldots\vee C^n) = \text{\textbf{False}}$, т.е. является несущественной. Поэтому
$$F = \forall_X A\rightarrow\bigl(\Phi\vee\exists_Y\exists_{Z_1}(B\&C_1\&\Psi\&\Phi_1)\vee\ldots\vee\exists_Y\exists_{Z_k}(B\&C_k\&\Psi\&\Phi_k))\bigr),$$
что при переводе в ПОФ-представление имеет вид совпадающий с $F'$.

Если $F$ начинается с квантора $\exists$, после перевода в язык исчисления предикатов получаем:
$$F = \exists_X A\&\Phi\&(\forall_Y B\rightarrow (\Psi\vee C\&\forall_{Z_1}(C_1\rightarrow\Phi_1)\&\ldots\&\forall_{Z_k}(C_k\rightarrow\Phi_k))).$$
Распишем выражение в скобке:
$$
\begin{array}{l}
\forall_Y B\rightarrow (\Psi\vee C\&\forall_{Z_1}(C_1\rightarrow\Phi_1)\&\ldots\&\forall_{Z_k}(C_k\rightarrow\Phi_k) =\\
\forall_Y ((\neg B'\vee\neg C^1\vee\ldots\vee\neg C^n\vee\Psi) \vee C^1\&\ldots\& C^n\&\forall_{Z_1}(C_1\rightarrow\Phi_1)\&\ldots\&\forall_{Z_k}(C_k\rightarrow\Phi_k)) =\\
\forall_Y ((\neg B'\vee\neg C^1\vee\ldots\vee\neg C^n\vee\Psi\vee C^1)\&\ldots\&(\neg B'\vee\neg C^1\vee\ldots\vee\neg C^n\vee\Psi\vee C^n)\&\\
(\neg B'\vee\neg C^1\vee\ldots\vee\neg C^n\vee\Psi\vee\forall_{Z_1}(C_1\rightarrow\Phi_1) )\&\ldots\&\\
(\neg B'\vee\neg C^1\vee\ldots\vee\neg C^n\vee\Psi\vee\forall_{Z_k}(C_k\rightarrow\Phi_k) )) = \\
\forall_Y (\text{\textbf{True}}\&\ldots\&\text{\textbf{True}}\&(\neg B\vee\Psi\vee\forall_{Z_1}(C_1\rightarrow\Phi_1) )\&\ldots\&(\neg B\vee\Psi\vee\forall_{Z_k}(C_k\rightarrow\Phi_k) )) = \\
\forall_Y\forall_{Z_1}(\neg(B\& C_1)\vee\Psi\vee\Phi_1)\&\ldots\&\forall_Y\forall_{Z_k}(\neg(B\& C_k)\vee\Psi\vee\Phi_k)
\end{array}
$$
Т.е. $$F = \exists_X A\&\Phi\&\forall_Y\forall_{Z_1}((B\& C_1)\rightarrow(\Psi\vee\Phi_1))\&\ldots\&\forall_Y\forall_{Z_k}((B\& C_k)\rightarrow(\Psi\vee\Phi_k)),$$
что при переводе в ПОФ-представление имеет вид совпадающий с $F'$.

Доказательство завершено.

Квантор $Q\colon C$, в формуле $F$ из правила сокращения, называется тривиальным квантором. Если в формуле $F$ из правила сокращения, корневым узлом является квантор $P_Y B$, то необходимо приветсти дерево к соответствующему виду, добавив новый корень, фиктивный квантор $Q\varnothing$. Правило сокращения следует применять к парам узлов, начиная от листьев дерева, и на ``прямых'' участках внутри дерева, которые не ведут к ветвлению. Применение правила приводящее к ветвлению следует избегать, если только получаемое дерево не будет состоять из меньшего количества узлов. Например, сокращение формулы из рассматриваемого выше примера приводит к следующему:
\begin{center}
\begin{tikzpicture}
\tikzset{grow'=right}
%\tikzset{execute at begin node=\strut}
%\tikzset{level distance=70pt}
\tikzset{level 1/.style={level distance=60pt}}
\tikzset{level 2/.style={level distance=40pt}}
\tikzset{level 3+/.style={level distance=110pt}}
\tikzset{every tree node/.style={align=left,anchor=west}}
\Tree [. $\forall x,y\colon\varnothing$ [. $\exists\varnothing$ [. $\forall z\colon S(x,y),I(z,x)$ $\exists\colon I(x,y)$ ] [. $\forall\varnothing$ [. $\exists z\colon I(z,x)$ $\forall\colon I(x,y)$ ] $\exists\colon S(x,y)$ ] ] ]
\end{tikzpicture}
\end{center}
В таком случае возможны два варианта, сократить либо квантор $\exists\varnothing$, либо $\forall\varnothing$, последнее ведет к большему количеству узлов дерева, т.к. ветку $\forall z\colon S(x,y),I(z,x) - \exists\colon I(x,y)$ придется скопировать при ветвлении. Окончательно, дерево выглядит следующим образом:

\begin{center}
\begin{tikzpicture}
\tikzset{grow'=right}
%\tikzset{execute at begin node=\strut}
%\tikzset{level distance=70pt}
\tikzset{level 1/.style={level distance=30pt}}
\tikzset{level 2/.style={level distance=130pt}}
\tikzset{level 3+/.style={level distance=160pt}}
\tikzset{every tree node/.style={align=left,anchor=west}}
\Tree [. $\exists\varnothing$ [. $\forall x,y,z\colon S(x,y),I(z,x)$ $\exists\colon I(x,y)$ ] [. $\forall x,y\colon\varnothing$ [. $\exists z\colon I(z,x)$ $\forall\colon I(x,y)$ ] $\exists\colon S(x,y)$ ] ]
\end{tikzpicture}
\end{center}

\section{Conclusion}
A new efficient algorithm for conversion of the predicate calculus language formulas to the language of positively constructed formulas (PCF's) and equivalent rules of reduction of the  PCFs is considered in this paper.  The algorithm preserves the original heuristic structure of knowledge represented by the formulas of the predicate calculus language.

One of the further development of the conversion module is an implementation of the algoritm in a new version of prover, adopted to syntax and properties of TPTP problem set library \cite{tptp}.  Another direction is to adapt the algorithm to the disjunctive language widely used in provers based on the resolution method, as most of the problems of the library are represented in the conjunctive normal forms.  The main problem here is to reconstruct the heuristically structure broken during scolemization and conversion (from first--order language) stages.  We already have a version of the presented algorithm for conjunctive normal form formulas, which have no existential variables.

\section{!ACQ!}
\label{sec:acq}



\begin{thebibliography}{9}
\bibitem{SNV1990} S.N.~Vassilyev \emph{Machine Synthesis of Mathematical Theorems}. The Journal of Logic programming, V.9, No.2--3, pp. 235--266,1990.
\bibitem{ICDS2000} S.N.~Vassilyev, A.K.~Zherlov, E.A.~Fedunov, B.E.~Fedosov \emph{Intelligent Control of Dynamic Systems}. Moscow, Russia: Fizmatlit, 2000. (in Russian)
\bibitem{jour2} A.V.~Davydov, A.A.~Larionov, E.A.~Cherkashin \emph{On the calculus of positively constructed formulas for automated theorem proving}. Automatic Control and Computer Sciences (AC\&CS), V.45, No.7, pp. 402--407, 2011.
\bibitem{mipro2013} E.A.~Cherkashin, A.A.~Larionov, A.V.~Davydov \emph{Calculus of Positively Constructed Formulas, its Features, Strategies and Implementation} // MIPRO 2013/CIS (36-th international convention on information and communication technology, electronics and microelectronics), 20-24 May 2013, Chroatia, Opatija, 2013. - pp. 1289--1295.
\bibitem{Bourbaki} N.~Bourbaki \emph{Theory of Sets}. Paris: Hermann, 1968.
\bibitem{tptp} G.~Sutcliffe \emph{The TPTP Problem Library and Associated Infrastructure. The FOF and CNF Parts, v3.5.0} // Journal of Automated Reasoning. V43, N4, pp.337--362.
\end{thebibliography}

\end{document}

%%% Local Variables:
%%% mode: latex
%%% TeX-master: t
%%% End:
